\documentclass{scrarticle}
\usepackage{hyperref}
\begin{document}
\section*{Code-Besprechung 12.03.2021}
\subsection*{Changelog}
\begin{itemize}
\item Kern\"anderungen (siehe scripts/pre/startScript*.m in git-repo \url{https://git.scc.kit.edu/franke/pyfemkit})
  \begin{itemize}
  \item auto updates via dofClass
  \item flags in classes and runNewton-script for update-procedures (idea: for further classes hopefully no need for adjustement of runNewton-script)
  \item (failed) catch objects via property of classes by creation of an instance
  \item baseFEClass (nodes, edof, globalFullEdof, globalNodesDof, dimension)
  \item solidSuperClass (qR, qN1, qN, vN1, vN) (derived from baseFEClass)
  \item clear/destruct all 'pre' Objects except continuumObjects
  \item function structure of runNewton-script
  \item speed up: save dofObject \(\rightarrow\) object search
\end{itemize}
\item proof of concept    
\begin{itemize}
  \item discrete Gradient Elementroutine
  \item solidThermoClass (proof of concept for auto updates)
  \item numerical Tangent (FD) function-based structure (parfor-loop not possible)
\end{itemize}
\item zus\"atzliche \"Anderungen
\begin{itemize}
    \item dofs \(\rightarrow\) dof (e.g. dofsObject \(\rightarrow\) dofObject)
  \item dirichletClass \(\rightarrow\) implement full features (arbritary dof, timeFunction: include nodal information)
\end{itemize}
\end{itemize}
\subsection*{TODO}
\begin{itemize}
  \item basic feature: universal numerical tangent (FD \& complex tangent)
  \item proof of concept: mixed finite elements \(\rightarrow\) solver/static condensation
 \item automatisches Testen mittels Testskript von Beginn an: CI (matlabrunner) in git checken
 \item python implementation 
\end{itemize}
\end{document}

%%% Local Variables:
%%% mode: latex
%%% TeX-master: t
%%% End:
